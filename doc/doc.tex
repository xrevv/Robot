\documentclass[11pt,titlepage,a4paper]{article}
\usepackage{setspace}

\usepackage[style=american]{csquotes}

\usepackage[backend=biber,style=numeric,sorting=none]{biblatex}
\addbibresource{bib.bib}

\usepackage{fancyhdr}
\setlength{\headheight}{14pt}
\pagestyle{fancy}

\lhead{}
\chead{Ramię robota}
\rhead{}
\lfoot{}
\cfoot{}
\rfoot{str. \thepage}

% Polski
\usepackage[]{polski} 
\usepackage[polish]{babel}

% Pierwszy akapit - wcięty
\usepackage[]{indentfirst}

% eps
\usepackage{graphicx}
\usepackage{subfigure}

\renewcommand*{\thesubsubsection}{}

\usepackage[dvipsnames]{xcolor}

\usepackage{hyperref}
\hypersetup{
    colorlinks,
    citecolor=black,
    filecolor=black,
    linkcolor=black,
    urlcolor=black
}

\hbadness = 1

\title{
\includegraphics[scale=0.75]{img/politechnika_sl_logo_bw_poziom_pl.eps}\\
\textbf{
Wydział Automatyki, Elektroniki i Informatyki}\\
\vspace*{1cm}
Projekt z Systemów Mikroprocesorowych\\
Ramię robota\\
}
% TODO Nie jestem pewnien sekcji
\author{Mateusz Siedliski i Radosław Tchórzewski\\
Rok akademicki 2022/2023, semestr 5, grupa 6, sekcja 2\\
\\
Kierujący pracą: dr inż. Jacek Loska} 
\date{Gliwice 2023}

\begin{document}

\onehalfspacing

\maketitle

\tableofcontents

\newpage

\section{\textcolor{red}{NOT FINAL} Wstęp}

Podczas studiowania na kierunku Automatyka i Robotyka można zauważyć zadziwiający brak fizycznych pomocy naukowych. Ten projekt ma na celu poprawę tej sytuacji choćby w niewielkim stopniu. W tym celu zaproponowano stworzenie ramienia robota (manipulatora). Ma on na celu pomoc studentom z wizualizacją koncepcji teoretycznych w prawdziwym święcie, a nie tylko w książkach, czy na ekranie komputera.

Projekt może posłużyć także do zachęcenia potencjalnych studentów podczas na przykład dni otwartych, czy wycieczek szkolnych.

Główną inspiracją projektu był film zamieszczony na platformie YouTube z kanału \enquote{How To Mechatronics} pod tytułem \enquote{DIY Arduino Robot Arm with Smartphone Control} \cite*{HTM_YT}.

Nasz projekt korzysta z tych samych technologii, aczkolwiek wszystkie elementy (model 3D, oprogramowanie mikrokontrolelera, kod aplikacji, schemat połączeń itd.) zostały przygotowane przez nas.

W ramach projektu stworzono dydaktyczny model 5 osiowego manipulatora z chwytakiem, zrealizowanego w technologii druku 3D, sterowany aplikacją na urządzenia z systemem Android.

\subsection{\textcolor{orange}{TODO} Cel i zakres projektu}

W tym punkcie opisać cel i zakres projektu oraz do czego jej wyniki mogą być wykorzystane. Określić ogólne wymagania sprzętowe i programowe potrzebne do realizacji projektu.

Przykładowe etapy projektu który zostanie wykonany i opisany dalej, mogą być następujące:
\begin{itemize}
    \item Określenie problemu i wykonanie do niego założeń.
    \item Analiza możliwych rozwiązań, wraz z kryteriami wyboru.
    \item Wybór na postawie założonych kryteriów wraz z uzasadnieniem takiego a nie innego rozwiązania.
    \item Wykonanie projektu zgodnie z wcześniejszymi założeniami.
    \item Uruchomienie, weryfikacji i przetestowanie sprzętu i aplikacji.
    \item Nakreślenie ewentualnych kierunków rozwoju projektu.
    \item Wnioski końcowe.
\end{itemize}

Ta część powinna zwierać od 1 do 2 stron.

\newpage

\section{\textcolor{red}{NOT FINAL} Harmonogram}

\subsection{\textcolor{green}{DONE} Harmonogram zatwierdzony}

\begin{enumerate}
    \item Projektowanie modelu fizycznego robota oraz jego druk w technologii 3D.
    \item Montaż mechaniczny oraz elektryczny.
    \item Tworzenie oprogramowania na mikrokontroler.
    \item Tworzenie aplikacji sterującej.
    \item Projektowanie oraz realizacja komunikacji między mikrokontrolerem, \\a aplikacją sterującą.
\end{enumerate}

\subsection{\textcolor{red}{NOT FINAL} Harmonogram wykonany}

\begin{enumerate}
    \item Projektowanie modelu fizycznego robota oraz jego druk w technologii 3D. Montaż mechaniczny oraz elektryczny.
    \item Tworzenie oprogramowania na mikrokontroler oraz aplikacji sterującej. Opracowanie protokołu komunikacji między mikrokontrolerem, a aplikacją sterującą.
    \item Doskonalenie projektu — debugowanie, poprawki mechaniczne
    \item Doskonalenie projektu — debugowanie, poprawki mechaniczne
    \item Doskonalenie projektu — debugowanie, poprawki mechaniczne
\end{enumerate}

\newpage

\section{\textcolor{red}{NOT FINAL} Kosztorys}

\begin{center}
    \begin{tabular}{|r|l|l|c|r|r|}
        \hline
        Lp. & Typ                          & Producent & Ilość & Cena    & Wartość  \\
        \hline
        1.  & Mikrokontroler Wemos D1 mini & Wemos     & 1     & 9,48 zł & 9,48 zł  \\
        2.  & Moduł Bluetooth HC-05        & SZYTF     & 1     & 9,40 zł & 9,40 zł  \\
        3.  & Micro Servo 9g SG90          & HWAYEH    & 3     & 3,97 zł & 11,91 zł \\
        4.  & Servo Mg996r                 & WAVGAT    & 3     & 12 zł   & 36 zł    \\
        5.  & Obudowa ramienia (druk 3D)   & n/d       & 1     & 20 zł   & 20 zł    \\
        6.  & Przewody                     & n/d       & n/d   & n/d     & 5 zł     \\
        \hline
        \multicolumn{6}{|r|}{Suma = 91,79 zł}                                       \\
        \hline
        \multicolumn{6}{|r|}{Ilość roboczogodzin}                                   \\
        \hline
    \end{tabular}
\end{center}

\textcolor{red}{TODO: Ilość roboczogodzin poświęconych na projekt.}

\newpage

\section{\textcolor{orange}{TODO} Urządzenie wraz z aplikacją}

Należy podać krótki opis celu który chcemy osiągnąć wykonując projekt.
Pozostałe punkty mają ułatwić napisanie projektu, ale nie są bezwzględnie wymagane. Powinno się je elastycznie dopasować do zrealizowanego projektu – SZCZEGÓLNIE nazwy podrozdziałów.
Ta część powinna zwierać od 20 do 25 stron.

\subsection{\textcolor{orange}{TODO} Określenie problemu}

Dokładne opisanie problemu. Wykonać schematy ideowe i blokowe, dołączyć ilustrujące problem rysunki itp. Wykonać założenia potrzebne do rozwiązania postawionego problemu.

Rysunek 1. To jest rysunek podstawowego bloku systemu.

Należy zawsze odwoływać się do tego co jest na Rysunek 1.

\subsection{\textcolor{orange}{TODO} Analiza rozwiązań}

W tym punkcie powinno się przedstawić jakie są możliwości rozwiązania problemu (urządzenia, technologie i produkty), wraz z określeniem kryteriów wyboru rozwiązania.

\subsection{\textcolor{orange}{TODO} Zaproponowane rozwiązanie}

W tym punkcie należy przedstawić wybrane rozwiązanie problemu, wraz z uzasadnieniem wyboru na postawie kryteriów z poprzedniego punktu.

\subsection{\textcolor{orange}{TODO} Wykonanie}

W tym punkcie opisać jak zostało wykonane urządzenie oraz zaprogramowana aplikacja, jakie zastosowano technologie, narzędzia do pisania, weryfikowania i testowania aplikacji. Nie należy umieszczać kodu programu, jedynie wybrane fragmenty pokazujące rozwiązanie niestandardowego problemu. Pełny kod programu należy przesłać emailem ew. dodatkowo umieścić w załączniku.
Jest to najważniejszy punkt w projekcie – proszę mu poświęcić jak najwięcej uwagi minimum 15-18 stron!!!

\subsection{\textcolor{orange}{TODO} Problemy w trakcie tworzenia sprzętu i aplikacji (niepotrzebne usunąć!)}

W tym punkcie należy przedstawić jakie były problemy przed którymi stanęli autorzy projektu w trakcie jego realizacji i jak je rozwiązali.

\newpage

\section{\textcolor{orange}{TODO} Podsumowanie}

W ostatnim punkcie opisać co zostało wykonane, jakiej części założeń nie wykonano i dlaczego. Co można zrobić, by dany projekt poprawić i w jakim kierunku może pójść dalszy rozwój tego projektu.

\newpage

\section{\textcolor{red}{NOT FINAL}  Literatura}

\printbibliography[heading=none]

\newpage

\section{\textcolor{orange}{TODO} Załączniki}

Na prezentację (obronę) projektu trzeba przygotować prezentację w PowerPoint lub Prezi trwającą ok 12-15 minut Zwykle jest to ok. 20 slajdów. Te slajdy muszą zawierać tytuł projektu, autora, prowadzącego. Następnie plan prezentacji, najważniejsze osiągnięcia , podsumowanie. Slajdy muszą być czytelne, należy umieszczać rysunki i schematy, także nie powinno się na nich umieszczać zbyt dużo tekstu (tekst powinien mieć ok. 20-24pkt). W trakcie lub po prezentacji powinno się zademonstrować wykonany sprzęt i oprogramowanie. Można również zaprezentować w zamian tego film z działania urządzenia.

Przykładowa zawartość załącznika:
Kody źródłowe (z komentarzami!!!) na CD spakowane.
Wysłane mailem zawartość projektu do prowadzącego.
Prezentacja.
Wszystkie opisy układów, programów, narzędzi stosowanych w projekcie itp.


\end{document}