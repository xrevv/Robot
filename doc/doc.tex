\documentclass[11pt,titlepage]{article}

\usepackage{fancyhdr}
\pagestyle{fancy}

\lhead{}
\chead{Ramię robota}
\rhead{}
\lfoot{}
\cfoot{}
\rfoot{str. \thepage}

\usepackage{geometry}
\geometry{
    a4paper,
    total={210mm,297mm},
    left=20mm,
    right=20mm,
    top=20mm,
    bottom=20mm,
}

% Polski
\usepackage[]{polski} 
\usepackage[polish]{babel}

% Pierwszy akapit - wcięty
\usepackage[]{indentfirst}

% Coś do matmy
\usepackage[]{amsfonts}
\usepackage[]{amsmath}


% Coś z formatowania
\usepackage{ragged2e}

% Tytuły sekcji na środku
% \usepackage{titlesec}
% \titleformat{\section}[block]{\Large\bfseries\filcenter}{}{1em}{}

% <=
\usepackage{amssymb}


% eps
\usepackage{graphicx}
\usepackage{subfigure}

\renewcommand*{\thesubsubsection}{}

\usepackage[numbered]{matlab-prettifier}
\lstset{
    literate={ą}{{\k{a}}}1
    {Ą}{{\k{A}}}1
    {ę}{{\k{e}}}1
    {Ę}{{\k{E}}}1
    {ó}{{\'o}}1
    {Ó}{{\'O}}1
    {ś}{{\'s}}1
    {Ś}{{\'S}}1
    {ł}{{\l{}}}1
    {Ł}{{\L{}}}1
    {ż}{{\.z}}1
    {Ż}{{\.Z}}1
    {ź}{{\'z}}1
    {Ź}{{\'Z}}1
    {ć}{{\'c}}1
    {Ć}{{\'C}}1
    {ń}{{\'n}}1
    {Ń}{{\'N}}1
}

% Dzielenie wyrazów
% \tolerance=1
% \emergencystretch=\maxdimen
% \hyphenpenalty=10000
% \hbadness=10000

\usepackage{hyperref}
\hypersetup{
    colorlinks,
    citecolor=black,
    filecolor=black,
    linkcolor=black,
    urlcolor=black
}

\title{
\textbf{Politechnika Śląska w Gliwicach\\
Wydział Automatyki, Elektroniki\\
i Informatyki}\\
\vspace*{1cm}
Projekt z Systemów Mikroprocesorowych\\
Ramię robota\\
}

\author{Mateusz Siedliski i Radosław Tchórzewski\\
Kierujący pracą: dr inż. Jacek Loska} 
\date{Gliwice 2022}

\begin{document}

\maketitle

\tableofcontents

\newpage

\section{Wstęp}

We wstępie należy umieścić krótką charakterystykę projektu oraz przedstawić do czego projekt ma służyć i jego ogólne założenia. Jest to opis ogólny wprowadzenie w tematykę projektu, funkcjonalny, bez wdawania się w szczegóły techniczne.

Ta część powinna zwierać od ok. 1 strony.

Całość projektu piszemy w formie bezosobowej – ZAPROPONOWANO, ZROBIONO, WYKONANO itp.

Każdy główny rozdział powinien rozpoczynać się od nowej strony (należy wstawić znak końca strony lub końca sekcji a nie wpisywać ciąg znaków końca linii).

Poszczególne fragmenty tekstu powinny być grupowane w postaci akapitów – oznaczanych przez wcięcie albo przez odstęp między akapitami. Należy używać czcionki 11pkt i odstępu interlinii 1,5.

Poziomów wcięcia w pracy powinno być maksymalnie 3.

\subsection{Cel i zakres projektu}

W tym punkcie opisać cel i zakres projektu oraz do czego jej wyniki mogą być wykorzystane. Określić ogólne wymagania sprzętowe i programowe potrzebne do realizacji projektu.

Przykładowe etapy projektu który zostanie wykonany i opisany dalej, mogą być następujące:
\begin{itemize}
    \item Określenie problemu i wykonanie do niego założeń.
    \item Analiza możliwych rozwiązań, wraz z kryteriami wyboru.
    \item Wybór na postawie założonych kryteriów wraz z uzasadnieniem takiego a nie innego rozwiązania.
    \item Wykonanie projektu zgodnie z wcześniejszymi założeniami.
    \item Uruchomienie, weryfikacji i przetestowanie sprzętu i aplikacji.
    \item Nakreślenie ewentualnych kierunków rozwoju projektu.
    \item Wnioski końcowe.
\end{itemize}

Ta część powinna zwierać od 1 do 2 stron.

\section{Harmonogram}

Przedstawić opis i założenia harmonogramu pracy z podziałem na dwie części:

\subsection{Harmonogram zatwierdzony}

Podać dokładnie taki harmonogram jaki został zatwierdzony na początkowych zajęciach.

\subsection{Harmonogram wykonany}

Harmonogram powinien obejmować faktycznie zrealizowany program projektu. Powinny także być przedstawione powody jego ewentualnej modyfikacji wraz z uzasadnieniem.

\section{Kosztorys}

W tym rozdziale należy umieścić w tabeli wszystkie elementy z których składa się projekt wraz nazwą, producentem, ceną jednostkową, ich liczbą oraz kosztem. Musi być podany całkowity koszt projektu.

Przykładowa tabela może wyglądać następująco

Na końcu tabeli należy podać sumę elementów.
Oraz ilość roboczogodzin poświęconych na projekt.

\section{Urządzenie wraz z aplikacją}

Należy podać krótki opis celu który chcemy osiągnąć wykonując projekt.
Pozostałe punkty mają ułatwić napisanie projektu, ale nie są bezwzględnie wymagane. Powinno się je elastycznie dopasować do zrealizowanego projektu – SZCZEGÓLNIE nazwy podrozdziałów.
Ta część powinna zwierać od 20 do 25 stron.

\subsection{Określenie problemu}

Dokładne opisanie problemu. Wykonać schematy ideowe i blokowe, dołączyć ilustrujące problem rysunki itp. Wykonać założenia potrzebne do rozwiązania postawionego problemu.

Rysunek 1. To jest rysunek podstawowego bloku systemu.

Należy zawsze odwoływać się do tego co jest na Rysunek 1.

\subsection{Analiza rozwiązań}

W tym punkcie powinno się przedstawić jakie są możliwości rozwiązania problemu (urządzenia, technologie i produkty), wraz z określeniem kryteriów wyboru rozwiązania.

\subsection{Zaproponowane rozwiązanie}

W tym punkcie należy przedstawić wybrane rozwiązanie problemu, wraz z uzasadnieniem wyboru na postawie kryteriów z poprzedniego punktu.

\subsection{Wykonanie}

W tym punkcie opisać jak zostało wykonane urządzenie oraz zaprogramowana aplikacja, jakie zastosowano technologie, narzędzia do pisania, weryfikowania i testowania aplikacji. Nie należy umieszczać kodu programu, jedynie wybrane fragmenty pokazujące rozwiązanie niestandardowego problemu. Pełny kod programu należy przesłać emailem ew. dodatkowo umieścić w załączniku.
Jest to najważniejszy punkt w projekcie – proszę mu poświęcić jak najwięcej uwagi minimum 15-18 stron!!!

\subsection{Problemy w trakcie tworzenia sprzętu i aplikacji (niepotrzebne usunąć!)}

W tym punkcie należy przedstawić jakie były problemy przed którymi stanęli autorzy projektu w trakcie jego realizacji i jak je rozwiązali.

\section{Podsumowanie}

W ostatnim punkcie opisać co zostało wykonane, jakiej części założeń nie wykonano i dlaczego. Co można zrobić, by dany projekt poprawić i w jakim kierunku może pójść dalszy rozwój tego projektu.

\section{Literatura}

Zawsze jest wymagane stosowanie odwołań do źródeł. Jeżeli skądkolwiek zostały zaczerpnięte rysunki, teksty, schematy, część kodu programu należy to umieścić w spisie literatury, a w tekście wstawić odnośniki w formie nr odnośnka np. odnośnik 2 to odwołanie do technologii pisania tekstów!!!

Nazwisko Imię, Tytuł, Wydawnictwo, Rok wydania.

Strona internetowa – Koniecznie podać datę dostępu.

\section{Załączniki}

Na prezentację (obronę) projektu trzeba przygotować prezentację w PowerPoint lub Prezi trwającą ok 12-15 minut Zwykle jest to ok. 20 slajdów. Te slajdy muszą zawierać tytuł projektu, autora, prowadzącego. Następnie plan prezentacji, najważniejsze osiągnięcia , podsumowanie. Slajdy muszą być czytelne, należy umieszczać rysunki i schematy, także nie powinno się na nich umieszczać zbyt dużo tekstu (tekst powinien mieć ok. 20-24pkt). W trakcie lub po prezentacji powinno się zademonstrować wykonany sprzęt i oprogramowanie. Można również zaprezentować w zamian tego film z działania urządzenia.

Przykładowa zawartość załącznika:
Kody źródłowe (z komentarzami!!!) na CD spakowane.
Wysłane mailem zawartość projektu do prowadzącego.
Prezentacja.
Wszystkie opisy układów, programów, narzędzi stosowanych w projekcie itp.


\end{document}